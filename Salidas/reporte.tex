\PassOptionsToPackage{unicode=true}{hyperref} % options for packages loaded elsewhere
\PassOptionsToPackage{hyphens}{url}
%
\documentclass[]{article}
\usepackage{lmodern}
\usepackage{amssymb,amsmath}
\usepackage{ifxetex,ifluatex}
\usepackage{fixltx2e} % provides \textsubscript
\ifnum 0\ifxetex 1\fi\ifluatex 1\fi=0 % if pdftex
  \usepackage[T1]{fontenc}
  \usepackage[utf8]{inputenc}
  \usepackage{textcomp} % provides euro and other symbols
\else % if luatex or xelatex
  \usepackage{unicode-math}
  \defaultfontfeatures{Ligatures=TeX,Scale=MatchLowercase}
\fi
% use upquote if available, for straight quotes in verbatim environments
\IfFileExists{upquote.sty}{\usepackage{upquote}}{}
% use microtype if available
\IfFileExists{microtype.sty}{%
\usepackage[]{microtype}
\UseMicrotypeSet[protrusion]{basicmath} % disable protrusion for tt fonts
}{}
\IfFileExists{parskip.sty}{%
\usepackage{parskip}
}{% else
\setlength{\parindent}{0pt}
\setlength{\parskip}{6pt plus 2pt minus 1pt}
}
\usepackage{hyperref}
\hypersetup{
            pdftitle={Reporte investigación},
            pdfauthor={Ignacio Becker Bozo},
            pdfborder={0 0 0},
            breaklinks=true}
\urlstyle{same}  % don't use monospace font for urls
\usepackage[margin=1in]{geometry}
\usepackage{color}
\usepackage{fancyvrb}
\newcommand{\VerbBar}{|}
\newcommand{\VERB}{\Verb[commandchars=\\\{\}]}
\DefineVerbatimEnvironment{Highlighting}{Verbatim}{commandchars=\\\{\}}
% Add ',fontsize=\small' for more characters per line
\usepackage{framed}
\definecolor{shadecolor}{RGB}{248,248,248}
\newenvironment{Shaded}{\begin{snugshade}}{\end{snugshade}}
\newcommand{\AlertTok}[1]{\textcolor[rgb]{0.94,0.16,0.16}{#1}}
\newcommand{\AnnotationTok}[1]{\textcolor[rgb]{0.56,0.35,0.01}{\textbf{\textit{#1}}}}
\newcommand{\AttributeTok}[1]{\textcolor[rgb]{0.77,0.63,0.00}{#1}}
\newcommand{\BaseNTok}[1]{\textcolor[rgb]{0.00,0.00,0.81}{#1}}
\newcommand{\BuiltInTok}[1]{#1}
\newcommand{\CharTok}[1]{\textcolor[rgb]{0.31,0.60,0.02}{#1}}
\newcommand{\CommentTok}[1]{\textcolor[rgb]{0.56,0.35,0.01}{\textit{#1}}}
\newcommand{\CommentVarTok}[1]{\textcolor[rgb]{0.56,0.35,0.01}{\textbf{\textit{#1}}}}
\newcommand{\ConstantTok}[1]{\textcolor[rgb]{0.00,0.00,0.00}{#1}}
\newcommand{\ControlFlowTok}[1]{\textcolor[rgb]{0.13,0.29,0.53}{\textbf{#1}}}
\newcommand{\DataTypeTok}[1]{\textcolor[rgb]{0.13,0.29,0.53}{#1}}
\newcommand{\DecValTok}[1]{\textcolor[rgb]{0.00,0.00,0.81}{#1}}
\newcommand{\DocumentationTok}[1]{\textcolor[rgb]{0.56,0.35,0.01}{\textbf{\textit{#1}}}}
\newcommand{\ErrorTok}[1]{\textcolor[rgb]{0.64,0.00,0.00}{\textbf{#1}}}
\newcommand{\ExtensionTok}[1]{#1}
\newcommand{\FloatTok}[1]{\textcolor[rgb]{0.00,0.00,0.81}{#1}}
\newcommand{\FunctionTok}[1]{\textcolor[rgb]{0.00,0.00,0.00}{#1}}
\newcommand{\ImportTok}[1]{#1}
\newcommand{\InformationTok}[1]{\textcolor[rgb]{0.56,0.35,0.01}{\textbf{\textit{#1}}}}
\newcommand{\KeywordTok}[1]{\textcolor[rgb]{0.13,0.29,0.53}{\textbf{#1}}}
\newcommand{\NormalTok}[1]{#1}
\newcommand{\OperatorTok}[1]{\textcolor[rgb]{0.81,0.36,0.00}{\textbf{#1}}}
\newcommand{\OtherTok}[1]{\textcolor[rgb]{0.56,0.35,0.01}{#1}}
\newcommand{\PreprocessorTok}[1]{\textcolor[rgb]{0.56,0.35,0.01}{\textit{#1}}}
\newcommand{\RegionMarkerTok}[1]{#1}
\newcommand{\SpecialCharTok}[1]{\textcolor[rgb]{0.00,0.00,0.00}{#1}}
\newcommand{\SpecialStringTok}[1]{\textcolor[rgb]{0.31,0.60,0.02}{#1}}
\newcommand{\StringTok}[1]{\textcolor[rgb]{0.31,0.60,0.02}{#1}}
\newcommand{\VariableTok}[1]{\textcolor[rgb]{0.00,0.00,0.00}{#1}}
\newcommand{\VerbatimStringTok}[1]{\textcolor[rgb]{0.31,0.60,0.02}{#1}}
\newcommand{\WarningTok}[1]{\textcolor[rgb]{0.56,0.35,0.01}{\textbf{\textit{#1}}}}
\usepackage{graphicx,grffile}
\makeatletter
\def\maxwidth{\ifdim\Gin@nat@width>\linewidth\linewidth\else\Gin@nat@width\fi}
\def\maxheight{\ifdim\Gin@nat@height>\textheight\textheight\else\Gin@nat@height\fi}
\makeatother
% Scale images if necessary, so that they will not overflow the page
% margins by default, and it is still possible to overwrite the defaults
% using explicit options in \includegraphics[width, height, ...]{}
\setkeys{Gin}{width=\maxwidth,height=\maxheight,keepaspectratio}
\setlength{\emergencystretch}{3em}  % prevent overfull lines
\providecommand{\tightlist}{%
  \setlength{\itemsep}{0pt}\setlength{\parskip}{0pt}}
\setcounter{secnumdepth}{0}
% Redefines (sub)paragraphs to behave more like sections
\ifx\paragraph\undefined\else
\let\oldparagraph\paragraph
\renewcommand{\paragraph}[1]{\oldparagraph{#1}\mbox{}}
\fi
\ifx\subparagraph\undefined\else
\let\oldsubparagraph\subparagraph
\renewcommand{\subparagraph}[1]{\oldsubparagraph{#1}\mbox{}}
\fi

% set default figure placement to htbp
\makeatletter
\def\fps@figure{htbp}
\makeatother


\title{Reporte investigación}
\author{Ignacio Becker Bozo}
\date{7/2/2020}

\begin{document}
\maketitle

\hypertarget{introducciuxf3n}{%
\subsection{Introducción}\label{introducciuxf3n}}

En base a las constantes problemáticas de inserción laboral juvenil en
las últimas décadas (CEPAL/OIJ 2004; CEPAL/OIT 2017), además de una
subutilización de las fuentes de datos disponibles para medir sus
condiciones, el presente proyecto busca aportar en una cuantificación de
este fenómeno, mediante el uso de la Encuesta Nacional de Empleo del
Instituto Nacional de Estadísticas.

Los procesos que reestructuraron la organización del empleo a mediados
de siglo XX, relacionados a los mecanismos de flexibilización de los
mercados laborales, junto a la precarización de la fuerza de trabajo,
han orquestado novedosos mecanismos de empleo y contratación, basados en
los contratos temporales, las jornadas parciales, la tercerización
productiva -como la subcontratación-, entre otros dispositivos propios
del empleo atípico (Neffa 2008; Stecher and Sisto 2019), que han
debilitado aún más el empleo, impidiendo la generación de vínculos
estables, protegidos y seguros para la integración social (Kaztman
2001).

Este desarrollo, sin embargo, no afecta a todos los grupos sociales en
la misma medida: el aumento de la flexibilidad supone una desregulación
dirigida, alterando y reformando en primera instancia los márgenes del
empleo (Atkinson 1984), para luego ampliarse al resto de la fuerza de
trabajo (O'Reilly et al. 2015). Los mismos autores señalan que este
proceso ha promovido la segmentación del mercado laboral, llevándose la
carga flexibilizadora aquellos empleados que son considerados atípicos,
o al margen del empleo formal. Varios autores dan cuenta cómo la
población joven es aquella que sufre estas condiciones en mayor medida
(Furlong 1990; Jacinto 2006; O'Reilly et al. 2015; Weller 2007), en el
denominado mercado laboral juvenil (o \emph{Youth Labor Market}
{[}YLM{]}, por su origen británico), el cual discute si estas
condiciones son propias de la juventud, o se debe a situaciones de paso
hacia el ingreso al mercado laboral (Ashton, Maguire, and Spilsbury
1990).

\hypertarget{objetivos-proyecto}{%
\section{Objetivos proyecto}\label{objetivos-proyecto}}

En este sentido, y dado el carácter acotado del presente proyecto (el
cual consta de solo un semestre), es que en el presente escrito se
propone los siguientes objetivos, entendiendo los procesos de
reestructuración y flexibilización de la fuerza de trabajo juvenil en
los últimos años.

\hypertarget{objetivo-general}{%
\subsection{Objetivo general}\label{objetivo-general}}

Caracterizar la temporalidad contractual de la población joven en Chile
en base al trimestre marzo-abril-mayo 2019 de la Encuesta Nacional de
Empleo.

\hypertarget{objetivos-especuxedficos}{%
\subsection{Objetivos específicos}\label{objetivos-especuxedficos}}

\begin{itemize}
\tightlist
\item
  Describir la temporalidad contractual de las y los jóvenes asalariados
  en Chile, según variables sociodemográficas (como sexo, edad, región
  de residencia) y educativa.
\item
  Identificar la temporalidad de la ocupación según las ramas de empleo
  de la población joven ocupada.
\item
  Caracterizar la posesión de contrato por variables sociodemográficas.
\end{itemize}

\hypertarget{resultados}{%
\section{Resultados}\label{resultados}}

A nivel de descriptivos preliminares, las variables a considerar son las
siguientes:

\begin{Shaded}
\begin{Highlighting}[]
\KeywordTok{dfSummary}\NormalTok{(ene, }\DataTypeTok{style =} \StringTok{"grid"}\NormalTok{)}
\end{Highlighting}
\end{Shaded}

\begin{verbatim}
## Warning in png(png_loc <- tempfile(fileext = ".png"), width = 150 *
## graph.magnif, : unable to open connection to X11 display ''
\end{verbatim}

\begin{verbatim}
## Warning in png(png_loc <- tempfile(fileext = ".png"), width = 150 *
## graph.magnif, : unable to open connection to X11 display ''
\end{verbatim}

\begin{verbatim}
## Warning in png(png_loc <- tempfile(fileext = ".png"), width = 150 *
## graph.magnif, : unable to open connection to X11 display ''
\end{verbatim}

\begin{verbatim}
## Warning in png(png_loc <- tempfile(fileext = ".png"), width = 150 *
## graph.magnif, : unable to open connection to X11 display ''

## Warning in png(png_loc <- tempfile(fileext = ".png"), width = 150 *
## graph.magnif, : unable to open connection to X11 display ''
\end{verbatim}

\begin{verbatim}
## Warning in png(png_loc <- tempfile(fileext = ".png"), width = 150 *
## graph.magnif, : unable to open connection to X11 display ''

## Warning in png(png_loc <- tempfile(fileext = ".png"), width = 150 *
## graph.magnif, : unable to open connection to X11 display ''
\end{verbatim}

\begin{verbatim}
## Warning in png(png_loc <- tempfile(fileext = ".png"), width = 150 *
## graph.magnif, : unable to open connection to X11 display ''
\end{verbatim}

\begin{verbatim}
## Data Frame Summary  
## ene  
## Dimensions: 107092 x 8  
## Duplicates: 52870  
## 
## +----+--------------------+--------------------------------------+-------------------------+----------------------+--------------------------------+----------+----------+
## | No | Variable           | Label                                | Stats / Values          | Freqs (% of Valid)   | Graph                          | Valid    | Missing  |
## +====+====================+======================================+=========================+======================+================================+==========+==========+
## | 1  | cae_especifico     | Código sumario de empleo específico  | Mean (sd) : 6.4 (7.8)   | 29 distinct values   | :                              | 107092   | 0        |
## |    | [haven_labelled]   |                                      | min < med < max:        |                      | :                              | (100%)   | (0%)     |
## |    |                    |                                      | 0 < 1 < 28              |                      | :                              |          |          |
## |    |                    |                                      | IQR (CV) : 12 (1.2)     |                      | :                              |          |          |
## |    |                    |                                      |                         |                      | :   .   . . . .                |          |          |
## +----+--------------------+--------------------------------------+-------------------------+----------------------+--------------------------------+----------+----------+
## | 2  | activ              | Condición de actividad               | Mean (sd) : 1.9 (1)     | 1 : 47921 (55.3%)    | IIIIIIIIIII                    | 86687    | 20405    |
## |    | [haven_labelled]   |                                      | min < med < max:        | 2 :  3504 ( 4.0%)    |                                | (80.95%) | (19.05%) |
## |    |                    |                                      | 1 < 1 < 3               | 3 : 35262 (40.7%)    | IIIIIIII                       |          |          |
## |    |                    |                                      | IQR (CV) : 2 (0.5)      |                      |                                |          |          |
## +----+--------------------+--------------------------------------+-------------------------+----------------------+--------------------------------+----------+----------+
## | 3  | r_p_rev4cl_caenes  | Rama de actividad económica de la    | Mean (sd) : 9.3 (5.7)   | 21 distinct values   |     :                          | 47921    | 59171    |
## |    | [haven_labelled]   | empresa o institución que le paga el | min < med < max:        |                      | :   :                          | (44.75%) | (55.25%) |
## |    |                    | sueldo o                             | 1 < 8 < 21              |                      | :   :       :                  |          |          |
## |    |                    |                                      | IQR (CV) : 9 (0.6)      |                      | :   : :     : :                |          |          |
## |    |                    |                                      |                         |                      | :   : : . . : : . :            |          |          |
## +----+--------------------+--------------------------------------+-------------------------+----------------------+--------------------------------+----------+----------+
## | 4  | ocup_form          | Ocupados según formalidad de la      | Mean (sd) : 0.6 (0.7)   | 0 : 59171 (55.2%)    | IIIIIIIIIII                    | 107092   | 0        |
## |    | [haven_labelled]   | ocupación                            | min < med < max:        | 1 : 33630 (31.4%)    | IIIIII                         | (100%)   | (0%)     |
## |    |                    |                                      | 0 < 0 < 2               | 2 : 14291 (13.3%)    | II                             |          |          |
## |    |                    |                                      | IQR (CV) : 1 (1.2)      |                      |                                |          |          |
## +----+--------------------+--------------------------------------+-------------------------+----------------------+--------------------------------+----------+----------+
## | 5  | sexo               | Sexo                                 | Min  : 1                | 1 : 51228 (47.8%)    | IIIIIIIII                      | 107092   | 0        |
## |    | [haven_labelled]   |                                      | Mean : 1.5              | 2 : 55864 (52.2%)    | IIIIIIIIII                     | (100%)   | (0%)     |
## |    |                    |                                      | Max  : 2                |                      |                                |          |          |
## +----+--------------------+--------------------------------------+-------------------------+----------------------+--------------------------------+----------+----------+
## | 6  | edad               | Edad de la persona                   | Mean (sd) : 38.4 (23)   | 105 distinct values  | . . :   : .                    | 107092   | 0        |
## |    | [numeric]          |                                      | min < med < max:        |                      | : : : : : :                    | (100%)   | (0%)     |
## |    |                    |                                      | 0 < 38 < 104            |                      | : : : : : : :                  |          |          |
## |    |                    |                                      | IQR (CV) : 38 (0.6)     |                      | : : : : : : : :                |          |          |
## |    |                    |                                      |                         |                      | : : : : : : : : .              |          |          |
## +----+--------------------+--------------------------------------+-------------------------+----------------------+--------------------------------+----------+----------+
## | 7  | region             | Región                               | Mean (sd) : 8.8 (4)     | 16 distinct values   |               :                | 107092   | 0        |
## |    | [haven_labelled]   |                                      | min < med < max:        |                      |               :                | (100%)   | (0%)     |
## |    |                    |                                      | 1 < 8 < 16              |                      |     . . . .   :                |          |          |
## |    |                    |                                      | IQR (CV) : 8 (0.5)      |                      | . : : : : :   :                |          |          |
## |    |                    |                                      |                         |                      | : : : : : : . : : :            |          |          |
## +----+--------------------+--------------------------------------+-------------------------+----------------------+--------------------------------+----------+----------+
## | 8  | cine               | Clasificación Internacional de Nivel | Mean (sd) : 5.8 (34.5)  | 1   :  4606 ( 4.3%)  |                                | 107092   | 0        |
## |    | [haven_labelled]   | Educacional (CINE)                   | min < med < max:        | 2   :  6317 ( 5.9%)  | I                              | (100%)   | (0%)     |
## |    |                    |                                      | 1 < 5 < 999             | 3   : 21004 (19.6%)  | III                            |          |          |
## |    |                    |                                      | IQR (CV) : 2 (6)        | 4   : 10974 (10.2%)  | II                             |          |          |
## |    |                    |                                      |                         | 5   : 38862 (36.3%)  | IIIIIII                        |          |          |
## |    |                    |                                      |                         | 6   :  9087 ( 8.5%)  | I                              |          |          |
## |    |                    |                                      |                         | 7   : 14991 (14.0%)  | II                             |          |          |
## |    |                    |                                      |                         | 8   :   947 ( 0.9%)  |                                |          |          |
## |    |                    |                                      |                         | 9   :   175 ( 0.2%)  |                                |          |          |
## |    |                    |                                      |                         | 999 :   129 ( 0.1%)  |                                |          |          |
## +----+--------------------+--------------------------------------+-------------------------+----------------------+--------------------------------+----------+----------+
\end{verbatim}

Puedes incorporar dentro del texto un resultado de una función de R,
como 2 o el valor mínimo de la \texttt{speed} en la tabla anterior: Min.
: 4.0 .

Existen distintas formas de imprimir tablas. Es importante distinguir si
queremos un formato final en PDF o HTML. En cualquiera de los dos casos,
recomiendo revisar el paquete \texttt{kableExtra} y los ejemplos que
tiene en
\href{https://haozhu233.github.io/kableExtra/awesome_table_in_html.html}{HTML}
y
\href{https://haozhu233.github.io/kableExtra/awesome_table_in_pdf.pdf}{PDF}.

También se pueden crear directamente \textbf{gráficos}, e.g:

\includegraphics{reporte_files/figure-latex/pressure-1.pdf}

Al usar \texttt{echo\ =\ FALSE} en el trozo de código evitamos que se
imprima el código de R y solo vemos el resultado, en cambio:

\begin{Shaded}
\begin{Highlighting}[]
\KeywordTok{plot}\NormalTok{(pressure)}
\end{Highlighting}
\end{Shaded}

\includegraphics{reporte_files/figure-latex/pressure2-1.pdf}

Nos muestra el código y el resultado.

Las referencias por \emph{default} quedan al final, así que solo tenemos
que dejar el título del apartado

\hypertarget{referencias}{%
\section*{Referencias}\label{referencias}}
\addcontentsline{toc}{section}{Referencias}

\hypertarget{refs}{}
\leavevmode\hypertarget{ref-ashton_restructuring_1990}{}%
Ashton, David, Malcolm Maguire, and Mark Spilsbury. 1990.
\emph{Restructuring the Labour Market: The Implications for Youth}.
Springer.

\leavevmode\hypertarget{ref-atkinson_manpower_1984}{}%
Atkinson, John. 1984. ``Manpower Strategies for Flexible
Organisations.'' \emph{Personnel Management} 16 (8): 28--31.

\leavevmode\hypertarget{ref-cepaloij_juventud_2004}{}%
CEPAL/OIJ. 2004. ``La Juventud En Iberoamérica: Tendencias Y
Urgencias.''

\leavevmode\hypertarget{ref-cepaloit_employment_2017}{}%
CEPAL/OIT. 2017. ``Employment Situation in Latin America and the
Caribbean: The Transition of Young People from School to the Labour
Market.'' 17. \url{https://repositorio.cepal.org//handle/11362/42251}.

\leavevmode\hypertarget{ref-furlong_labour_1990}{}%
Furlong, Andy. 1990. ``Labour Market Segmentation and the Age
Structuring of Employment Opportunities for Young People.'' \emph{Work,
Employment and Society} 4 (2): 253--69.

\leavevmode\hypertarget{ref-jacinto_estrategias_2006}{}%
Jacinto, Claudia. 2006. ``Estrategias Sistémicas Y Subjetivas de
Transición Laboral de Los Jóvenes En Argentina. El Papel de Los
Dispositivos de Formación Para El Empleo.'' \emph{Revista de Educación}
341: 57--79.

\leavevmode\hypertarget{ref-kaztman_seducidos_2001}{}%
Kaztman, Rubén. 2001. ``Seducidos Y Abandonados: El Aislamiento Social
de Los Pobres Urbanos.'' \emph{Revista de La CEPAL}.

\leavevmode\hypertarget{ref-neffa_informalidad_2008}{}%
Neffa, J. C. 2008. ``La Informalidad, La Precariedad Y El Empleo No
Registrado En La Provincia de Buenos Aires.'' \emph{La Plata: Ministerio
de La Provincia de Buenos Aires}.

\leavevmode\hypertarget{ref-oreilly_five_2015}{}%
O'Reilly, Jacqueline, Werner Eichhorst, András Gábos, Kari
Hadjivassiliou, David Lain, Janine Leschke, Seamus McGuinness, Lucia
Mỳtna Kureková, Tiziana Nazio, and Renate Ortlieb. 2015. ``Five
Characteristics of Youth Unemployment in Europe: Flexibility, Education,
Migration, Family Legacies, and EU Policy.'' \emph{Sage Open} 5 (1):
2158244015574962.

\leavevmode\hypertarget{ref-stecher_trabajo_2019}{}%
Stecher, Antonio, and Vicente Sisto. 2019. ``Trabajo Y Precarización
Laboral En El Chile Neoliberal.'' In \emph{"Hilos Tensados". Para Leer
El Octubre Chileno}, edited by Kathya Araujo. Colección IDEA. Santiago
de Chile: IDEA.

\leavevmode\hypertarget{ref-weller_insercion_2007}{}%
Weller, Jürgen. 2007. ``La Inserción Laboral de Los Jóvenes:
Características, Tensiones Y Desafíos.'' \emph{Revista de La CEPAL} 92.

\end{document}
